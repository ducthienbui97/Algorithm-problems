% Fill in the template below. Here are some LaTeX quick tips:
% - %, {, }, $, _, ^, ~, \ and & are all special characters. Generally
%   precede them by a \ to get the literal character, although
%   \backslash gives a \ and \~{} gives a ~.
% - % is a comment character.
% - Surround maths with $'s (for example, $a+(b^c)$). LaTeX takes care
%   of correct spacing. You can get <= and >= with \le and \ge in maths
%   mode. _ and ^ give subscripts and superscripts. If the subscript or
%   superscript is more than one character, put it inside braces {}.
% - To get a ..., use \dots.
% - To get an opening single quote, use ` and not '
% - For double quotes, use `` to open and '' to close. LaTeX turns these
%   into double quotes.
% - A blank line starts a new paragraph.
% - To get a hyphen, use - (e.g. half-baked). To get a range, use -- (e.g.
%   1--10). To get a dash use --- (e.g. don't do that --- it will be taken
%   care of).
% - See the template Scoring section for how to use bullets. For a numbered
%   list, replace itemize with enumerate.
% - To make large numbers more readable use \ in math mode, as a thousands
%   separator. See the template Constraints section for an example.
% - Use "\ " (without quotes) after after full stops in abbreviations. This
%   tells LaTeX not to include a larger space used between sentences.
\documentclass{saco}
\begin{document}
\maketitle

\section{Introduction}
% Introduce the problem as a real-life situation.

Fred the Manic Storekeeper has decided to enter the mining industry. He is eyeing a large plot of land which he believes holds great riches below the surface.
However, he cannot decide precisely where to dig the mine.

\section{Task}
% Describe what task your program must perform.

Fred has partitioned the land into cells. For each cell, he has determined how much profit he can expect to make by mining at that position.
Some cells are resource rich and so give a large profit, while others are mainly dirt and so would result in a net loss.

Given a grid representing the profits obtainable from mining each of the cells (in thousands of rands), determine the rectangular region that would
yield the greatest total profit, were Fred to build his mine there.

\textbf{Implementation Note}: The \texttt{max} function in Python is very slow. If you are using Python and require this functionality, it may be necessary to implement the comparisons yourself to achieve top marks.

\section{Example}
% Provide a concrete example.

Suppose Fred provides us with the following grid.
\begin{table}[h!bt]
\begin{center}
\begin{tabular}{lll}
 1 & 1 & -4 \\
 4 & -1 & 3 \\
 -1 & 3 & 0
\end{tabular}
\end{center}
\end{table}

By choosing the lower $2$x$3$ rectangle, Fred can achieve a profit of $4-1+3-1+3+0 = 8$ or $\mathrm{R}8\,000$.
This choice is the best possible.

\inputformat
% Describe the input format. Be specific about whitespace. Note that
% the command above outputs the filename, so you should not mention
% it. Put variables into maths mode.
% Example:
% The first line of the input contains two space-separated integers, $N$
% and $T$. The next $N$ lines each contain four space-separated integers,
% $x_1, y_1, x_2$ and $y_2$. The lower-left and upper-right corner of the
% rectangles are at $(x_1, y_1)$ and $(x_2, y_2)$.

% This command pastes in the actual sample input file.
\sampleinput

\outputformat
% Describe the output format, as with the input format.

% This command pastes in the result of running the sample input through
% the sample solution.
\sampleoutput


\section{Constraints}
% Specify any constraints on the input (and possibly output) data. Make
% sure that all variables are constrained, even if the constraint is
% simply that it will fit into a signed 32-bit integer. It should
% generally be possible to write a solution that will get a correct
% answer for all legal data, not just the chosen test cases. Don't
% forget to set lower bounds, either here or by using words like
% "positive integer" in the input format. Use \, as a thousands
% as in the example below.
%
% Example:
\begin{itemize}
\item $1 \le N,M \le 190$
\item Each grid entry lies between $-1\,500$ and $1\,500$
\item At least one entry will be positive.
\end{itemize}

Additionally, in 30\% of the test cases:
\begin{itemize}
\item $N,M \le 20$
\end{itemize}

And in 60\% of the test cases:
\begin{itemize}
\item $N,M \le 50$
\end{itemize}

% Fill in constraints that will be met by half of the test cases. There
% should exist a simple solution that will solve these 50% constraints.

% This command will paste in the time limit from the options file.
\timelimit (Python: 10 seconds)

% This command will paste in the detailed feedback section if enabled.
\feedback

\section{Scoring}
% Provide an explicit scoring algorithm. Remember that you have to use
% \% to get a % sign. Specify what happens to fractional scores (if
% they can occur), since they must be converted to integers (usually
% rounded down).

A correct solution will score 100\%, whereas an incorrect solution will score 0\%.

\end{document}
