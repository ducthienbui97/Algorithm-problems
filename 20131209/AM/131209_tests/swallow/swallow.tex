% Fill in the template below. Here are some LaTeX quick tips:
% - %, {, }, $, _, ^, ~, \ and & are all special characters. Generally
%   precede them by a \ to get the literal character, although
%   \backslash gives a \ and \~{} gives a ~.
% - % is a comment character.
% - Surround maths with $'s (for example, $a+(b^c)$). LaTeX takes care
%   of correct spacing. You can get <= and >= with \le and \ge in maths
%   mode. _ and ^ give subscripts and superscripts. If the subscript or
%   superscript is more than one character, put it inside braces {}.
% - To get a ..., use \dots.
% - To get an opening single quote, use ` and not '
% - For double quotes, use `` to open and '' to close. LaTeX turns these
%   into double quotes.
% - A blank line starts a new paragraph.
% - To get a hyphen, use - (e.g. half-baked). To get a range, use -- (e.g.
%   1--10). To get a dash use --- (e.g. don't do that --- it will be taken
%   care of).
% - See the template Scoring section for how to use bullets. For a numbered
%   list, replace itemize with enumerate.
% - To make large numbers more readable use \ in math mode, as a thousands
%   separator. See the template Constraints section for an example.
% - Use "\ " (without quotes) after after full stops in abbreviations. This
%   tells LaTeX not to include a larger space used between sentences.
\documentclass{saco}
\begin{document}
\maketitle

\section{Introduction}
Bruce and Carl live in the same area and constantly send each other messages.
However, recently the local internet exchange went down and both have lost
internet access.

But all is not lost, as both have a large reserve of messenger swallows
(African, specifically) and coconuts! Unfortunately, the length of messages one
can write on a coconut is severely limited. Therefore, they wish to compress
the messages they send.

Thankfully, foreseeing this eventuality, both Bruce and Carl have agreed on a
compression algorithm (described below). Your task (other than carving the
messages on the coconuts) is to come up with a program to
optimally compress the messages.

\section{Compression Algorithm}
The messages you will be compressing consist purely of uppercase letters (from
A to Z). The compression algorithm works by encoding repeated subsequences as a
number (indicating the number of repetitions) and the subsequence. For instance,
\texttt{\'TESTTESTTESTTEST\'} would get encoded as \texttt{'4(TEST)'}.

But due to the notation we have chosen, we can recursively encode subsequences.
For instance, \texttt{CATDOGDOGCATDOGDOG} could be compressed as
\texttt{2(CAT2(DOG))}.

More formally, the decompression algorithm is, where $S$ is a compressed string:
\begin{itemize}
  \item If $S$ consists of a single uppercase letter, then it decompresses to
        $S$.
  \item If $S$ can be split into two compressed strings $T$ and $Q$, which
        decompress to $T'$ and $Q'$ respectively, then $S$ decompresses to
        $T'Q'$ ($T'$ followed by $Q'$).
  \item If $S$ is a compressed string in the form $k(Q)$, where $k$ is a
        decimal encoded integer and $Q$ is a compressed string, then $S$
        decompresses to $k$ repetitions of $Q'$, where $Q'$ is the
        decompression of $Q$.
\end{itemize}

\section{Task}
Given $N$ and a message of length $N$, you have to output the shortest possible
encoding of the message in the compression format.

If there are multiple possible ways of encoding the message you can output any
one of them.

\section{Example}
% Provide a concrete example.

\inputformat
The first line contains a single integer, $N$. The next line contains $N$
uppercase letters.

\sampleinput

\outputformat
A single line containing the shortest compression.

\sampleoutput

\section{Constraints}
\begin{itemize}
  \item $1 \leq N \leq 100$
\end{itemize}

Additionally, in 50\% of the test cases:
\begin{itemize}
  \item $1 \leq N \leq 13$
\end{itemize}

\timelimit

\feedback

\section{Scoring}
A correct solution will score 100\% while an incorrect solution will score
0\%.

\end{document}
